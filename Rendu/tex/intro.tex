\section{Introduction}

\lipsum[1-2]
\todo{Don't forget to put a real introduction here.}
%

%%%%% intro de flo %%%%%%

%Déterminer quelles entités taxonomiques constituent des espèces distinctes est un des objectifs premier de la taxonomie mais est également d’une importance primordiale pour comprendre à la fois l’origine des espèces, le processus de spéciation, et leur futur probable, qui détermine les mesures de conservation qui sont éventuellement à prendre pour les protéger.
%	Si délimiter des espèces vivant en sympatrie est généralement une tâche facile, les complexes d’espèces cryptiques ayant des distributions allopatriques présentent souvent plus de difficulté (Coyne & Orr 2004). En effet, dans ces derniers cas les critères qui servent généralement à établir la présence d’isolement reproductif (absence de flux de gènes ou  d’individus hybrides entre lignées, etc.) ne peuvent pas être mesurés objectivement (Fujita et al. 2012). Pourtant, délimiter des espèces cryptiques distribuées en allopatrie permet d’étudier le rôle d’un des mécanismes de spéciation majeurs : l’isolement géographique (Mayr 1942). Dans certains cas où la dynamique de l’environnement abiotique au cours du temps est bien connue, de telles études permettent même de caractériser précisément l’influence de l’environnement sur la divergence évolutive entre lignées (Avise et al. 1987).

%	Dans le cadre de ce stage, nous tenterons de comprendre comment la dynamique du relief dans les Alpes (i.e. les phénomènes d’orogénèse, d’érosion et de glaciation) a contribué à la divergence d’espèces au sein d’un groupe de plantes de haute montagne : Primula sect. Auricula Scott subsect. Erythrodrosum Pax (ci-dessous, clade Hirsuta). Ce groupe comprend traditionnellement six espèces proches de P. hirsuta All. mais une étude récente a proposé de re-circonscrire P. pedemontana en fusionnant trois espèces, tout en suggérant qu’un taxon distinct existe dans le massif des Ecrins en France (Boucher et al. 2016). Nos objectifs de recherche seront les suivants :
%	1) Reconstruire les relations phylogénétiques entre taxons du clade Hirsuta et dater leurs divergences
%	2) Délimiter les taxons qui méritent le rang d’espèce au sein de ce clade et en particulier au sein du complexe de P. pedemontana s.l.
%	3) Comparer statistiquement différents scénarios afin de comprendre comment la dynamique des reliefs Alpins a influencé la divergence des espèces du clade Hirsuta

%	Pour cela, nous utiliserons des données de séquençage haut-débit obtenues grâce à la technique hyRAD (Suchan et al. 2016) . Ces données comprennent plusieurs milliers de SNPs indépendants pour 22 individus appartenant aux six espèces du clade Hirsuta actuellement reconnues (Boucher et al. 2016). Nous les analyserons d’abord avec des approches phylogénétiques standard comme le logiciel RAxML (Stamatakis 2014) et des avancées nouvelles permettant de dater des phylogénies inférées grâce à des SNPs (Stange et al. 2018). Afin de délimiter des espèces le plus objectivement possible à l’aide des données moléculaires disponibles, nous utiliserons la panoplie variée des techniques de la taxonomie moléculaire, incluant des techniques de clustering génétique mais aussi d’autre basées sur le coalescent (Fujita et al. 2012, Carstens et al. 2013, Leaché et al. 2014). Enfin, nous aurons recours à l’inférence ABC pour comparer explicitement différents scénarios de spéciation (Knwoles 2009, Cornuet et al. 2014).

%	Cette étude devrait nous aider à mieux comprendre l’influence de la dynamique du relief sur l’évolution de la flore des Alpes. Elle permettra également d’établir plus fermement le statut taxonomique du nouveau taxon de Primula découvert dans le massif des Ecrins, qui reste à ce jour un mystère (http://www.ecrins-parcnational.fr/actualite/lenigme-de-la-primevere-du-valgaudemar), et éventuellement d’envisager des mesures de conservation.