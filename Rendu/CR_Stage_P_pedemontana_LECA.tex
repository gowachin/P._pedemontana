\documentclass[12pt,a4paper,notitlepage,colorinlistoftodos]{article}
%%%%%%%%%%%%%%%%%%%%%%%%%%%%%%%%%%%%%%%%%%%%%%%%%%%%%%%%%%%%%%%%%%%%%
% Template pour rendus Master
%%%%%%%%%%%%%%%%%%%%%%%%%%%%%%%%%%%%%%%%%%%%%%%%%%%%%%%%%%%%%%%%%%%%%%
\usepackage[utf8]{inputenc} %encodage
\usepackage[square,sort,comma,numbers]{natbib} % bibliography
\setcitestyle{authoryear,open={(},close={)}}
\renewcommand{\bibsection}{}

\usepackage[french]{babel} % langue

%mise en page générale
\usepackage{geometry}
%\geometry{a4paper} % format de feuille
\geometry{top=2.5cm, bottom=2.5cm, left=2.5cm, right=2.5cm} %marges
\usepackage{mathptmx} % Police Times si compilateur pdfLatex
\linespread{1.5} % interligne
\usepackage{fancyhdr} %en tete et pied de page
\usepackage{lastpage} 
\pagestyle{plain} 

\usepackage{lscape} % page en landscape

\usepackage{hyperref,url} % lien cliquables
\usepackage{lipsum} %Lorem ipsum

\usepackage{wrapfig} %position d'images dans le texte
\usepackage{graphicx, subcaption, setspace, booktabs, wrapfig}

\usepackage[table]{xcolor}

\usepackage{caption}
\DeclareCaptionType{annexe}[Annexe][Liste d'annexes] % rajout pour captions annexes
\DeclareCaptionType{web}[Web][Sites Web] % rajout pour mettre des captions web

\usepackage[disable]{todonotes} % notes et commentaires
%\usepackage[disable]{todonotes} % pour supprimer les commentaires lors de la compil

\usepackage[export]{adjustbox}


%%%%%%%%%%%% skip an all paragraphe, between this bornes %%%%%%%%%%%%%%%%%%
%\iffalse
%\fi

%%%%%%%%%%%%%%%%%%%%%%%%%% Code R   %%%%%%%%%%%%%%%%%%%%%%%%%%%%%%%%%%%%%%
\usepackage{listings}
\usepackage{color}
%http://latexcolor.com/ 
\definecolor{codegray}{rgb}{0.5,0.5,0.5}
\definecolor{cerulean}{rgb}{0.0, 0.48, 0.65}
\definecolor{beaublue}{rgb}{0.95, 0.95, 0.95}
\definecolor{amber}{rgb}{1.0, 0.25, 0.0}
\definecolor{indiagreen}{rgb}{0.07, 0.53, 0.03}
\definecolor{number}{rgb}{0.01, 0.01, 0.01}

\lstset{language = R,
    basicstyle=\footnotesize,
    breaklines=true,
    keepspaces=true,
    firstnumber=1,
    numbers=left, % where line-numbers; possible values (none, left, right)
    numbersep=5pt,  % how far the line-numbers are from the code
    numberstyle=\color{number},
    deletekeywords={_,/,C,troll,approx,min},
    backgroundcolor=\color{beaublue},   
    commentstyle=\color{indiagreen},
    keywordstyle=\color{amber},
    stringstyle=\color{cerulean}
    }
    
%\begin{lstlisting}
%  %%%% put the R code here %%%%
%\end{lstlisting}

%%%%%%%%%%%%%%%%%%%%%%%%%% LATEX DIFF %%%%%%%%%%%%%%%%%%%%%%%%%%%%%%%%%%%%%%%%%%%
% use terminal: latexdiff ancientfile.tex newfile.tex > revisionfile.tex

%DIF UNDERLINE PREAMBLE %DIF PREAMBLE
\RequirePackage[normalem]{ulem} %DIF PREAMBLE
\RequirePackage{color}\definecolor{RED}{rgb}{1,0,0}\definecolor{BLUE}{rgb}{0,0,1} %DIF     PREAMBLE
\providecommand{\DIFadd}[1]{{\protect\color{blue}\uwave{#1}}} %DIF PREAMBLE
\providecommand{\DIFdel}[1]{{\protect\color{red}\sout{#1}}}                      %DIF PREAMBLE
%DIF SAFE PREAMBLE %DIF PREAMBLE
\providecommand{\DIFaddbegin}{} %DIF PREAMBLE
\providecommand{\DIFaddend}{} %DIF PREAMBLE
\providecommand{\DIFdelbegin}{} %DIF PREAMBLE
\providecommand{\DIFdelend}{} %DIF PREAMBLE
%DIF FLOATSAFE PREAMBLE %DIF PREAMBLE
\providecommand{\DIFaddFL}[1]{\DIFadd{#1}} %DIF PREAMBLE
\providecommand{\DIFdelFL}[1]{\DIFdel{#1}} %DIF PREAMBLE
\providecommand{\DIFaddbeginFL}{} %DIF PREAMBLE
\providecommand{\DIFaddendFL}{} %DIF PREAMBLE
\providecommand{\DIFdelbeginFL}{} %DIF PREAMBLE
\providecommand{\DIFdelendFL}{} %DIF PREAMBLE
%DIF END PREAMBLE EXTENSION ADDED BY LATEXDIFF

%%%%%%%%%%%%%%%%%%%%%%%%%%% nouvelles commandes spécifique au doc %%%%%%%%%%%%%%%%%%
\newcommand{\clade}[1]{clade `#1'}

\DeclareRobustCommand{\rchi}{{\mathpalette\irchi\relax}}
\newcommand{\irchi}[2]{\raisebox{\depth}{$#1\chi$}}

%%%%%%%%%%%%%%%%%%%%%%%%%%%%%%%%%%%%%%%%%%%%%%%%%%%%%%%%%%%%%%%%%%%%%%
% Page de garde
%%%%%%%%%%%%%%%%%%%%%%%%%%%%%%%%%%%%%%%%%%%%%%%%%%%%%%%%%%%%%%%%%%%%%%
\title{\textbf{Délimitation d'espèces au sein du complexe de plantes des Alpes, \textit{Primula pedemontana s.l.}}}
\author{Maxime Jaunatre, Master 1 BEE Grenoble}
\date{1 Avril - 31 Mai 2019  |  Soutenance : juin 2019 }

\renewcommand*\contentsname{Table des matières}
\begin{document}

\begin{titlepage} %Page de garde!!
%logos de page de garde
\begin{figure}
\noindent
\begin{minipage}{0.5\textwidth}
\centering
\includegraphics[width=0.5\linewidth,left]{fig/UGA.jpg}
\end{minipage}
\begin{minipage}{0.5\textwidth}
\centering
\includegraphics[width=0.5\linewidth,right]{fig/leca.jpg}
\end{minipage}
\end{figure}
\maketitle

\noindent
\begin{minipage}{1.3in}
\textbf{\underline{Enseignant référent:}} \\
Éric Coissac
\end{minipage}
\hfill
\begin{minipage}{1.3in}
\textbf{\underline{Tuteur de stage:}} \\
Florian Boucher
\end{minipage}
\hfill
\begin{minipage}{1.3in}
\textbf{\underline{Équipe:}} \\
DivAdapt
\end{minipage}

%image sympas
\begin{figure}[h]
\begin{center}
\includegraphics[scale=0.1]{fig/Primula_hirsuta_Grand_Chat_longistyle.JPG}
\end{center}
\end{figure}
\thispagestyle{empty}

{\let\thefootnote\relax\footnote{{ \textit{Primula hirsuta},  Crédit photo : Florian Boucher}}}

%%%%%%%%%%%%%%%%%%%%%%%%%%%%%%%%%%%%%%%%%%%%%%%%%%%%%%%%%%%%%%%%%%%%%%
% Abstract
%%%%%%%%%%%%%%%%%%%%%%%%%%%%%%%%%%%%%%%%%%%%%%%%%%%%%%%%%%%%%%%%%%%%%%
\newpage
\begin{abstract}


La dynamique des Alpes au Quaternaire a permis l'apparition d'une flore très diverse. Cependant, cette diversité porte les traces de multiples événements génétiques difficiles à différencier. Cette étude propose de réanalyser un jeu de données phylogénétique sur la sous-section \textit{Erythrodrosum}, située au sein d'un large groupe de fleur de montagnes endémiques du système alpin : la section \textit{Auricula} du genre \textit{Primula}. Pour cela, les SNPs produits par séquençage à haut débit avec capture par sondes ARN, et alignés sur un génome de référence, sont re-triés pour obtenir un jeu de données analysable par les outils de génétique des populations. Ce nouveau jeu de donnée permet de valider les résultats précédemment établis sur la structure génétique de \textit{P. pedemontana}, proposant une révision taxonomique, ainsi que la confirmation d'un flux de gène entre cette espèce et \textit{P. hirsuta} dans le massif des Écrins. Finalement, cette étude préliminaire recentre les futures campagnes d'échantillonnage dans les massifs des Écrins et de la Vanoise, dans le but de clarifier l'impact des cycles de glaciations sur la génétique des fleurs de montagnes.


\thispagestyle{empty}

%%%%%%%%%%%%%%%%%%%%%%%%%%%%%%%%%%%%%%%%%%%%%%%%%%%%%%%%%%%%%%%%%%%%%%
% Todo list pour travail en cours
%%%%%%%%%%%%%%%%%%%%%%%%%%%%%%%%%%%%%%%%%%%%%%%%%%%%%%%%%%%%%%%%%%%%%%
%\newpage
%\listoftodos

\end{abstract}

%%%%%%%%%%%%%%%%%%%%%%%%%%%%%%%%%%%%%%%%%%%%%%%%%%%%%%%%%%%%%%%%%%%%%%
% Table des matières
%%%%%%%%%%%%%%%%%%%%%%%%%%%%%%%%%%%%%%%%%%%%%%%%%%%%%%%%%%%%%%%%%%%%%%
\newpage
\tableofcontents
\thispagestyle{empty}

\end{titlepage}

\newpage
%%%%%%%%%%%%%%%%%%%%%%%%%%%%%%%%%%%%%%%%%%%%%%%%%%%%%%%%%%%%%%%%%%%%%%
% BODY
%%%%%%%%%%%%%%%%%%%%%%%%%%%%%%%%%%%%%%%%%%%%%%%%%%%%%%%%%%%%%%%%%%%%%%
%Consignes pour rapport et soutenance du stage M1

%Le rapport
%Il doit comporter 12 pages (+/1page) figures comprises, sommaire et références biblio en plus, annexes déconseillées. Moins de 20 références bibliographiques, 12 lignes de résumé au dos de la couverture.
%Le plan à adopter a priori est « Introduction / Matériel et méthodes / Résultats / Discussion » à moins qu’il ne soit pas adapté à votre travail (c’est rarement le cas).
%Format : TIMES 12 points, avec un interligne 1,5. Le rapport peut être rédigé en Français ou en Anglais.
%Les personnes qui font un stage long et concentrent leur rapport sur la mise au point de la méthodologie (c'est à dire n'ayant pas de résultats au moment de la soutenance) sont encouragées à expliquer ce contexte dans un court préambule d'une dizaine de lignes maximum. Les membres de jury sont informés de l'éventualité d'une telle situation.
%Le rapport de stage est à rendre à votre responsable pédagogique de stage pour le :
%Mardi 28 mai 2018 à 12 h au plus tard
%Une version papier adressée par courrier ou déposée au secrétariat du LECA. Une version pdf envoyée par mail au tuteur de stage et au responsable du M1.
%L’envoi uniquement au format électronique peut se faire à titre exceptionnel (stage à l’étranger nécessitant un long délai d’acheminement – contacter par email votre responsable pédagogique avant).

%La soutenance
%Modalités : 10 min de présentation + 5 min de questions devant un jury de 3 personnes. Le maître de stage est invité à venir écouter la soutenance s’il le souhaite.
%Format de présentation : fichier pdf ou ppt sur clé USB. Les soutenances auront lieu la semaine du 3 Juin 2019. Un planning de soutenance vous sera communiqué fin mai. Les étudiants effectuant un stage long à l’étranger et ne rentrant pas en France peuvent soutenir par visioconférence.
%Contacter votre responsable pédagogique par email mi mai pour fixer les modalités.

%Fiche d’évaluation
%La fiche d'évaluation doit être donnée à votre encadrant en début de stage. Elle sera renvoyée par email par le maître de stage au plus tard le 1 er juin 2019.
%Attention : pour les stages longs à l'étranger prévoyez l'éventualité d'un retour à Grenoble en cas de rattrapage (seconde session d'examen deuxième quinzaine de juin première quinzaine de juillet).

\section{Introduction}

\lipsum[1-2]
\todo{Don't forget to put a real introduction here.}
%

%%%%% intro de flo %%%%%%

%Déterminer quelles entités taxonomiques constituent des espèces distinctes est un des objectifs premier de la taxonomie mais est également d’une importance primordiale pour comprendre à la fois l’origine des espèces, le processus de spéciation, et leur futur probable, qui détermine les mesures de conservation qui sont éventuellement à prendre pour les protéger.
%	Si délimiter des espèces vivant en sympatrie est généralement une tâche facile, les complexes d’espèces cryptiques ayant des distributions allopatriques présentent souvent plus de difficulté (Coyne & Orr 2004). En effet, dans ces derniers cas les critères qui servent généralement à établir la présence d’isolement reproductif (absence de flux de gènes ou  d’individus hybrides entre lignées, etc.) ne peuvent pas être mesurés objectivement (Fujita et al. 2012). Pourtant, délimiter des espèces cryptiques distribuées en allopatrie permet d’étudier le rôle d’un des mécanismes de spéciation majeurs : l’isolement géographique (Mayr 1942). Dans certains cas où la dynamique de l’environnement abiotique au cours du temps est bien connue, de telles études permettent même de caractériser précisément l’influence de l’environnement sur la divergence évolutive entre lignées (Avise et al. 1987).

%	Dans le cadre de ce stage, nous tenterons de comprendre comment la dynamique du relief dans les Alpes (i.e. les phénomènes d’orogénèse, d’érosion et de glaciation) a contribué à la divergence d’espèces au sein d’un groupe de plantes de haute montagne : Primula sect. Auricula Scott subsect. Erythrodrosum Pax (ci-dessous, clade Hirsuta). Ce groupe comprend traditionnellement six espèces proches de P. hirsuta All. mais une étude récente a proposé de re-circonscrire P. pedemontana en fusionnant trois espèces, tout en suggérant qu’un taxon distinct existe dans le massif des Ecrins en France (Boucher et al. 2016). Nos objectifs de recherche seront les suivants :
%	1) Reconstruire les relations phylogénétiques entre taxons du clade Hirsuta et dater leurs divergences
%	2) Délimiter les taxons qui méritent le rang d’espèce au sein de ce clade et en particulier au sein du complexe de P. pedemontana s.l.
%	3) Comparer statistiquement différents scénarios afin de comprendre comment la dynamique des reliefs Alpins a influencé la divergence des espèces du clade Hirsuta

%	Pour cela, nous utiliserons des données de séquençage haut-débit obtenues grâce à la technique hyRAD (Suchan et al. 2016) . Ces données comprennent plusieurs milliers de SNPs indépendants pour 22 individus appartenant aux six espèces du clade Hirsuta actuellement reconnues (Boucher et al. 2016). Nous les analyserons d’abord avec des approches phylogénétiques standard comme le logiciel RAxML (Stamatakis 2014) et des avancées nouvelles permettant de dater des phylogénies inférées grâce à des SNPs (Stange et al. 2018). Afin de délimiter des espèces le plus objectivement possible à l’aide des données moléculaires disponibles, nous utiliserons la panoplie variée des techniques de la taxonomie moléculaire, incluant des techniques de clustering génétique mais aussi d’autre basées sur le coalescent (Fujita et al. 2012, Carstens et al. 2013, Leaché et al. 2014). Enfin, nous aurons recours à l’inférence ABC pour comparer explicitement différents scénarios de spéciation (Knwoles 2009, Cornuet et al. 2014).

%	Cette étude devrait nous aider à mieux comprendre l’influence de la dynamique du relief sur l’évolution de la flore des Alpes. Elle permettra également d’établir plus fermement le statut taxonomique du nouveau taxon de Primula découvert dans le massif des Ecrins, qui reste à ce jour un mystère (http://www.ecrins-parcnational.fr/actualite/lenigme-de-la-primevere-du-valgaudemar), et éventuellement d’envisager des mesures de conservation.


\section{Matériels et méthodes}

\subsection{Échantillonnage}

Les données génétiques utilisés lors de cette étude ont été produite dans le cadre de l'étude précédente de \citet{Boucher2016a}. Il s'agit d'un jeu de donnée composé de 90 individus des espèces composant la section \textit{Auricula}, au sein du genre \textit{Primula}. Ces individus ont été prélevés à travers les Alpes entre avril et septembre 2014. L'identification taxonomique des individus a été réalisé sur le terrain, mais un individu a du être réattribué après analyse génétique au taxon \textit{P. hirsuta}.

Les SNPs sont issus de séquençage haut débit via hyRAD \citep{Suchan2016}. Cette technique permet de génotyper le long du génome malgré des mutations sur les sites de restrictions. En effet les enzymes de restrictions sont trop sensibles à la mutation d'un nucléotide, tandis que les sondes ARN peuvent s'hybrider sur des sites plus nombreux sur le génome. La nécessité de capturer des sites malgré une faible variation provient du niveau interspécifique de l'étude, qui pose l'hypothèse que les mutations peuvent se placer sur les sites de restrictions et ainsi limiter leur capture par simple séquençage ddRAD.

%We employ a newly developed next-generation-sequencing protocol, involving sequence capture with RAD probes, and map reads to the reference genome of Primula veris to obtain DNA matrices with thousands of SNPs

%Total DNA was extracted from silica-dried leaves using a DNeasy Plant Mini kit (Qiagen, Hilden, Germany) following the manufacturer’s instructions. DNA quality was visualized on 0.8% agarose gels and quantity was assessed using a QuBit 2.0 fluorom- eter (2.0, Life Technologies, Carlsbad, CA, USA). Genomic DNA was converted into RAD-Capture genotyping-by-sequencing libraries (SNPsaurus, LLC), a new protocol aimed at harnessing the wide genomic spectrum of RAD-sequencing while reducing the amount of missing data in interspecific datasets. Briefly, a double-digest RAD library was created from 100 ng of a pool of genomic DNA containing a diverse set of individuals from sect. Auricula (belong- ing to P. allionii, P. apennina, P. auricula, P. glutinosa, and P. minima; see Table A.1). The pooled DNA was digested with PstI-HF and MfeI- HF (NEB) and ligated to complementary adapters that allowed the resulting amplified fragments to be converted to biotinylated RNA baits. Fragments with inserts roughly 100–350 bp in size were iso- lated by gel extraction from a portion of the ligated product prior to amplification and the in vitro transcription reaction to create the RNA baits. Shotgun sequencing libraries were prepared from the 90 study samples using 5 ng each in a 1/10th Nextera (Illumina, Inc) reaction with unique dual-indexes to distinguish the individu- als. The samples were pooled and size-selected for insert sizes roughly 170–370 bp. The pooled libraries were then used in two successive overnight hybridizations to the biotinylated bait library, followed by capture using DynabeadsÒ MyOneTM Streptavidin C1 magnetic beads (Thermo Fisher) and amplification. The final cap- tured products were sequenced in a single 150 bp NextSeq 500 High Output run at the Genomics and Cell Characterization Core of the University of Oregon.

\subsection{Bioinformatique}

Pour chacun des individus, l'information consiste en une séquence de SNPs appellés par Freebayes, exporté sous format VCF. Les analyses ont été porté sur deux jeux de données différents car filtrés sous des seuils différents.
Le premier jeu de donnée ('\verb|m30_-q20_mincov20|') est issus de filtres très strict, avec un score de qualité (Phred) requis de 30 et une couverture minimale de 20 par site. Afin de ne pas biaiser l'analyse par des seuils favorisant les régions conservées, le second jeu de données ('\verb|m13_-q20_mincov10|') est quant à lui produit avec un Phred minimal de 13 et une couverture de 10. A partir de ces séquences, les SNPs ont été isolés par Freebayes, avec un score de Phred minimal de 20 et un support de lecture de 30\% minimum par allèle.

%parler du pipeline
A partir des deux jeux de donnée initiaux, un pipeline est établis pour générer divers ensembles de données. Cette automatisation a permit entre autre de pouvoir évaluer l'effet des seuils posés au fur et à mesure de l'analyse. Les fonctions sont rassemblées en un package R hébergé sur Github (lien web \ref{github})
Dans un premier temps, le fichier initial est traité sous Rstudio \citep{RTeam2017}, avec la fonction \verb|subset_reorder|, qui permet de reconstruire le fichier en ne gardant que les individus souhaité dans l'ordre indiqué. La fonction suivante \verb|rare|, permet de trier les allèles considérés comme étant présents dans un trop faible pourcentage des individus. Ces allèles rares sont écartés du jeu de donnée et le loci pour l'individu présentant cet allèle rare est considéré comme une donnée manquante. Cette étape permet également de supprimer les lectures avec de multiples allèles variants qui sont reconnus comme des artefacts des algorithmes utilisés pour appeler les SNP.  %We then filtered variants in a very conservative way by removing all called multiple nucleotide polymorphisms, indels and complex variants. These types of variants are known to often be artifacts generated by the algorithms used for short read mapping (see https://www. broadinstitute.org/gatk/guide/best-practices.php).
Suite aux deux tris précédents, il y a donc des loci pour lesquels tout les individus portent la même information. Afin de ne garder que les loci polymorphiques, une fonction \verb|clean| est donc appelée à la fin de la fonction \verb|rare|, pour supprimer les locis monomorphiques.
Les loci sont aussi filtré sur le score de qualité (\verb|QUAL|) indiqué dans le vcf, qui correspond a la confiance dans l'assignation de l'allèle variant. %In the context of variant calling, Phred-scaled quality scores can be used to represent many types of probabilities. The most commonly used in GATK is the QUAL score, or variant quality score. It is used in much the same way as the base quality score: the variant quality score is a Phred-scaled estimate of how confident we are that the variant caller correctly identified that a given genome position displays variation in at least one sample
Une fonction de \verb|tri| est appelé pour supprimer les loci puis individus présentant trop de données manquantes selon les seuils posés.
Enfin, afin de limiter le déséquilibre de liaisons, les sites sont filtrés selon leurs positions sur les contigs, où une distance minimale \verb|n| doit être prise en compte entre deux sites d'un même contig pour que le second site soit conservé.


Afin de pouvoir utiliser les fichiers triés sous divers format, la sous-sélection d'individus et de loci est enregistrée sous quatre formats : \verb|.vcf|, \verb|.str|, \verb|.geno|, \verb|.snp|. La transformation d'un format en un autre se fait respectivement au moyen d'un script bash, par l'utilisation du software PGDSpider \citep{Lischer2012}, le package LEA \citep{Frichot2015} et un script R. La production de ces fichiers est inscrite dans le pipeline par la fonction \verb|files|.

L'ensemble de ces fonctions sont indépendantes mais peuvent être appellées dans le bon ordre au moyen de la fonction \verb|dataset|, qui prend en entrée un vecteur de noms d'individus, un \verb|csv| contenant les assignations aux populations, le fichier \verb|vcf| original, le nom des fichiers de sortie et les seuils précisés au-dessus.


Chaque seuil est choisis après vérification que la décision n'impacte que peu les résultats de diversité génétique.
Le nombre d'individus étant faible, il faut optimiser le nombre de SNP par individus car cela permet de limiter la perte d'information et d'atteindre les mêmes résultats qu'attendus avec un échantillon plus vaste. \citep{Nazareno2017}


\subsection{Génétique des populations}

Les analyses de génétique des populations ont été réalisés sous Rstudio \citep{RTeam2017} avec différents packages. Le package adegenet a été utilisé pour inférer les fst, le package LEA pour étudier la structure de la population, package Introgress pour mesurer les introgressions entre P. pedemontana et P. hirsuta.

admixture aussi a peut etre été utilisé pour ça.

\subsection{Inférences bayesiennes}

Afin de caractériser l'admixture probable entre le taxon des Écrins et \textit{P. hirsuta}, une approche par approximate Bayesian computation (ABC) a été réalisée sur le logiciel DIYABC \cite{Cornuet2014}. Ce logiciel permet de simuler des jeux de données selon divers scénarios, en échantillonnant des paramètres entre des priors définis. Les scénarios sont ensuite classés selon les probabilités a posteriori d'observer notre jeu de donnée initial selon les scénarios proposés. Seuls quelques scénarios ont été étudiés ici, en prenant en compte le fait qu'ils sont à chaque fois supportés par peu d'individus. Les priors sont également proposés dans un grand interval et selon une distribution uniforme, les temps de divergence entre populations et taille de population n'ayant pas été étudiés sur le terrain.
%with the approximate Bayesian computation (ABC) approach implemented in the software DIYABC (Cornuet et al. 2014). In short, a large number of simu- lated data sets were produced under each tested scenario by sampling parameter values into predefined prior dis-tributions, and an estimation of each scenario posterior probability was calculated as a function of the similarity between several summary statistics of simulated data sets and observed data using a logistic regression proce-dure. These posterior probabilities enable a ranking of scenarios and allow for the identification of the most probable evolutionary history. We only considered five evolutionary scenarios (Fig. 2), based on previously pub-lished studies dealing with the evolution of the complex (Porter et al. 1994; Wiemers 1998; Schmitt & Besold 2010) and our own hypothesis based on the morphometric and phylogenetic analyses (see Results). Scenario design and simulations. A uniform distribution with a large interval was chosen for each prior increas-ing the need in computation time but overcoming the lack of knowledge on population sizes, divergence times and admixture rate (see Table S2 Supporting information). Only one approximate C. arcania /C. gardetta divergence time of 1.5–4 million of years was available from Kodandaramaiah & Wahlberg (2009). For each scenario, a total of one million data sets were sim-ulated and the posterior probability was computed by performing a logistic regression on the 1% of simulated data closest to the observed data set (Cornuet et al.2008, 2010). Summary statistics used for observed/sim- ulated data sets comparisons are the mean gene diver-sity and Fst across all loci and Nei’s distances among populations. Gene diversity was chosen because allelic richness is directly influenced by hybridization; the genetic differentiation (Fst) and distance (Nei’s) among taxa were used because these indexes reflect the degree of differentiation between populations and are highly relevant to compare evolutionary histories. Assessing the ‘goodness of fit’ of the model. Once the most probable scenario was identified, different posterior anal-yses were carried out to evaluate the trustfulness of the procedure. First, to be sure of the choice of the scenario, 500 new data sets were simulated with each historical model creating pseudo-observed data sets for which pos-terior probabilities were re-estimated with the same pro-cedure described above. Type I and type II errors were evaluated by measuring, respectively, the fraction of data sets simulated under the best scenario that were assigned to other scenarios and the fraction of data sets simulated under other scenarios that were assigned to the best sce- nario. Second, we wanted to know whether our selected model is able to produce data sets similar to the observed one. This confidence in the model is evaluated by esti-mating the similarity between simulated (1000 simula-tions) and real data sets using summary statistics different than the ones used for model choice (an impor-tant precaution, see Cornuet et al. 2010). For each sum-mary statistic, a P-value is estimated by ranking the observed value among the values obtained with simu-lated data sets, and a principal component analysis was also performed to check visually the position of the observed data in relation to the data sets generated from the posterior predictive distribution. Posterior distribution of parameters and their precision. It was possible to evaluate the posterior distributions of parameters for our selected scenario using a local linear regression on the 1% of simulations closest to the observed data set (Cornuet et al. 2010). These distribu-tions gave an idea of the most probable value (median) and the ‘approximation’ (width of the distribution) for each historical model. A precision of these estimations was evaluated by computing the median of the absolute error divided by the true parameter value of the 500 pseudo-observed data sets simulated under the selected scenario using the median of the posterior distribution as point estimate (Cornuet et al. 2010).


\section{Résultats}

\subsection{Tri du jeu de données}

Le jeu de donnée initial contient 175799 \textit{loci}. Pour le jeu de donnée utilisé au niveau de la sous-section \textit{Erythrodrosum}, 2078 \textit{loci} ont été conservés pour les analyses, avec les seuils suivants : allèle rare 5\%, Phred de 20, une distance de 10kb minimum entre deux \textit{loci} et 5\% des individus sans information par \textit{locus}.
Pour le jeu de donnée sur le \clade{Hirsuta}, 1851 \textit{loci} ont été conservés pour les analyses, avec les mêmes seuils.

Le changement de taille du jeu de donnée est du à un nombre plus petit d'individus dans le second cas, qui rend des \textit{loci} monomorphes. Le seuil le plus strict est le seuil sur les données manquantes par \textit{locus}, qui ne conserve que  respectivement 7.05\% et 7.07\% du jeu de donnée. 

\subsection{Sous-section \textit{Erythrodrosum}}

Dans un premier temps, le $F_{st}$ général de la sous-section \textit{Erythrodrosum} donne une valeur de 0.15, ce qui appuie l'hypothèse la présence d'une structure génétique. Cette structure étais attendue étant donné qu'il s'agit de plusieurs espèces à l'échelle des Alpes, mais pour autant il est important de remarquer que la valeur reste faible.
% Erythrodrosum
%            pop        Ind
% Total 0.1556567 -0.1016958
% pop   0.0000000 -0.3047961
\todo[color=blue!20]{proposer une explication? peut etre parce qu'il y a une sous structure? 
quid du Fis $<$0}

\begin{wrapfigure}{r}{0.30\textwidth}
	\vspace{-20pt}
	\begin{center}
	\includegraphics[width=0.29\textwidth]{fig/topologie.png}
	\end{center}
	\caption{\textbf{Topologie d'\textit{Erythrodrosum} par Neighbor-joining}, réalisé à partir de la matrice de distances $F_{st}$ par paires de populations.}
    \label{topologie}
\end{wrapfigure}

Les $F_{st}$ par paires entre populations sont également tous supérieurs à zero (valeurs comprises entre 0.093 et 0.214), et la topologie par neigbhor-joining rejoins celle décrite précédemment (Figure \ref{topologie}).
%\begin{wrapfigure}{r}{0.30\textwidth}
%	\vspace{-40pt}
%	\begin{center}
%	\includegraphics[width=0.29\textwidth]{fig/BIC_clustering_Erythro.png}
%	\end{center}
%	\caption{\textbf{cluster} cluster}
%    \label{cluster}
%\end{wrapfigure}
Cette structure est par ailleurs confirmée par le clustering réalisé par adegenet sans a-priori de populations. Ainsi, le nombre K de cluster avec le critère d'information bayésien le plus faible (BIC =   119.90) est atteins pour K=3. Pour cette valeur de K, les groupes sont corrélés à la géographie, avec un \clade{est-alpin} composé de \textit{P. daonensis} et \textit{P. villosa}, un clade composé de l'espèce \textit{P. hirsuta} et enfin \textit{P. pedemontana s.l.}.
% K=1      K=2      K=3      K=4      K=5      K=6      K=7      K=8      K=9     K=10     K=11     K=12     K=13     K=14     K=15     K=16     K=17     K=18     K=19     K=20 
% 121.8305 120.3166 119.9053 120.8310 122.0690 123.3748 124.7359 125.7829 126.8601 127.9004 128.7918 129.0472 129.3280 129.0943 128.6408 127.7302 125.9209 123.1317 117.1152 104.5158 
%BIC
\begin{wrapfigure}{r}{0.5\textwidth}
	\vspace{-20pt}
	\begin{center}
    \includegraphics[width=0.49\textwidth]{fig/DAPC.png}
    \caption{\textbf{Analyse en composante principale discriminante de la sous-section \textit{Erythrodrosum} Pax.} Les groupes à priori sont les populations échantillonnées.}
    \end{center}
    \label{DAPC}
    \vspace{-20pt}
\end{wrapfigure}

Ce clustering est plus facilement visualisable sous la forme d'une analyse en composante principale comme dans la figure \ref{DACP}. Les deux premières composantes montrent ce clustering avec quelques précisions de plus. Ainsi, \textit{P. pedemontana s.l.}. semble bien plus robuste que le \clade{est-alpin}, avec des individus plus proches. Cependant, la population des Écrins n'est pas aussi groupée que le reste des populations composant \textit{P. pedemontana s.l.}, et s'en éloigne vers \textit{P. hirsuta}.

\subsection{Clade `Hirsuta'}

La structuration du \clade{Hirsuta} pour différents K permet de voir différentes informations (figure \ref{structure}. Pour K=2, une séparation est déjà nette entre deux clades, conformément aux résultats précédents, entre \textit{P. pedemontana s.l.} et \textit{P. hirsuta}. Cette séparation présente néanmoins une légère trace d'admixture entre les individus des Écrins et \textit{P. hirsuta}. L'individu de \textit{P. hirsuta} présentant un peu d'admixture a été échantillonnée en Belledonne, un massif proche des Écrins (figure \ref{carte}).

\begin{wrapfigure}{r}{0.50\textwidth}
	\vspace{-40pt}
	\begin{center}
	\includegraphics[width=0.49\textwidth]{fig/structure_hirsuta.png}
	\end{center}
	\caption{\textbf{Structuration du \clade{Hirsuta} par sNMF.} Les K choisis vont de 2 à 5, avec 20 répétition par K et choix du meilleur run sur critère de cross-entropy.}
    \label{structure}
    \vspace{-30pt}
\end{wrapfigure}

Pour K=3, les individus des Écrins sont isolés et une admixture entre ces individus et les autres individus de \textit{P. pedemontana s.l.}. Cette structure retrouvée ici reflète ce qui étais observé sur la DAPC, où les individus des Écrins se regroupaient avec \textit{P. pedemontana s.l.} tout en étant excentré.

Pour K=4, \textit{P. pedemontana s.l.} se retrouve éclaté avec les différentes populations échantillonnées dans les divers massifs. Cependant, les deux espèces \textit{P. apennina} et \textit{P. cottia} sont toujours regroupées, même si cet ensemble n'est pas robuste. En effet, pour K=5, c'est \textit{P.hirsuta} qui se retrouve scindé en deux avec d'un côté l'individu des Pyrénées et de l'autre l'ensemble des individus. Ici la structure de \textit{P. pedemontana s.l.} est plus ambigue, même si les individus des Écrins sont toujours isolés.

%Pour les analyses sur DIYABC, aucun scénario proposé ne permet de simuler un jeu de donnée proche des données réelles. Le scénario observé le plus proche des données réelles étais celui d'un polytomie où coalescent tout les taxons assigné à l'espèce \textit{P. pedemontana}), puis un événement plus ancien de coalescence avec \textit{P. hirsuta}. Ne pouvant conclure si cette polytomie est réelle ou alors un artefact issu d'un manque de résolution (soft poolytomie), les analyses n'ont pas été plus loin avec cette outil et aucun résultat ne peut être présenté.

\subsection{Admixture}

Le test d'admixture entre les différentes populations de \textit{P. pedemontana s.l.} et \textit{P. hirsuta}, avec \textit{P. daonensis} en outgroup confirment l'admixture entre le taxon des Écrins et \textit{P. hirsuta}. En effet, le D estimé est supérieur à 0 avec une moyenne de 0.102. A contrario, le test ne permet pas d'estimer un D différent de 0 pour les autres populations, ce qui reflète les observations précédentes en sNMF.


\begin{table}
\begin{tabular}{ccccccc}\\\toprule  
P1 & P2 & P3 & Outgroup & D & p-valeur \\ \midrule
\textit{P. pedemontana} & \textit{P. apennina} & \textit{P. hirsuta} & \textit{P. daonensis} & -0.027 & 0.212 \\
\textit{P. pedemontana} & \textit{P. cottia} & \textit{P. hirsuta} & \textit{P. daonensis} & -0.033 & 0.0772 \\ \midrule
\textit{P. pedemontana} & Taxon des Écrins & \textit{P. hirsuta} & \textit{P. daonensis} & \textbf{0.082} & \textbf{5.761.10$^{-5}$} \\
\textit{P. apennina} & Taxon des Écrins & \textit{P. hirsuta} & \textit{P. daonensis} & \textbf{0.109} & \textbf{6.465.10$^{-7}$} \\
\textit{P. cottia} & Taxon des Écrins & \textit{P. hirsuta} & \textit{P. daonensis} & \textbf{0.116} & \textbf{1.237.10$^{-8}$} \\ \bottomrule
\end{tabular}
\caption{\textbf{Test d'admixture par ABBA-BABA}. La p-valeur est estimée à partir d'un bootstrap sur les \textit{loci} et recalcul de la valeur D. Un D supérieur à 0 indique une admixture entre P2 et P3.}
\label{ABBA}
\end{table}

Cependant, il est important de noter que ce test ne permet pas d'estimer quelle population de \textit{P. pedemontana s.l.} s'est admixté avec \textit{P. hirsuta}, vu que le résultat est positif quelque soit la population sélectionnée en P1. Ce résultat permet par contre de proposer deux hypothèses : la proximité génétique entre ces trois espèces ou alors une admixture passée avant séparation de ces populations sur des massifs isolés.

\todo[inline,color=blue!20]{totale réécriture de la partie introgress, article a creuser davantage}

%\begin{wrapfigure}{r}{0.30\textwidth}
%	\vspace{-40pt}
%	\begin{center}
%\missingfigure[figwidth=6cm,figheight= 6cm]{cline}
%		\includegraphics[width=0.29\textwidth]{file_name}
%	\end{center}
%	\caption{\textbf{cline} hirsuta pedemontana s.l.}
%    \label{cline}
%    \vspace{-20pt}
%\end{wrapfigure}


\section{Discussion}



\todo{To do}
% LIMITATIONS
Une grande limitation de cet étude provient de l'échantillonnage des populations.
 Ainsi, si l'échantillonnage actuel est suffisant pour les précédentes analyses phylogénétiques, la génétique des populations requiert davantage d'informations par populations.
 La plupart des analyses de génétiques des populations sont basées sur les variations de fréquences alléliques au sein des populations.
 La faible taille d'échantillon vient donc biaiser fortement ce prérequis, car plus la taille de l'échantillon est faible, moins l'estimation de la fréquence allélique représente la fréquence réelle au sein de la population.
 De fait, les potentielles structures génétiques des espèces étudiées ne peuvent être étudiées avec précision.
 Or, dans le cadre de cette étude, différentes structures génétiques sont superposée pour les populations étudiées.
 Par exemple la population de \textit{P. apennina} échantillonnée ici présente une structure génétique à l'échelle du massif des Apennins \citep{Crema2009}, mais représente aussi un partie de la structure des populations d'Alpes de l'ouest.
 Enfin, cette population doit représenter une entité structurelle pour la sous-section Erythrodrosum à l'échelle de l'arc Alpin.
 Si l'on ne considère que deux individus séquencés pour cette population, il est donc impossible d'avancer des analyses robustes pour les échelles les plus fines de structure.
 L'étude est donc de fait limité à la plus grande échelle possible, sans possibilité de résolution à l'échelle des Alpes de l'ouest.
 Il est d'ailleurs fort probable que ce soit ce manque de résolution qui ai pénalisé les analyses par inférence bayésienne.


% PERSPECTIVES
Un meilleur échantillonnage permettrais d'avoir suffisamment d'information génétique pour observer plus finement les statistiques F des populations.
 Plus d'échantillons permettrais également de pouvoir utiliser pleinement DIYABC et inférer une taille de population ou alors des paramètres historiques pour dater les événements biologiques.
 Enfin, un plus grand échantillon pourrais résoudre l'introgression entre \textit{P. hirsuta} et le taxon des Écrins, afin de déterminer le sens de cette introgression et quelle quantité du génome a été modifié par cet événement, à l'aide d'un test du même type que \verb|ABBA-BABA| mais avec P3 scindé en deux populations \citep{Eaton2015}. 

Autre aspect du rééchantillonnage,la délimitation des population au niveau géographique est un prérequis a un bon échantillonnage génétique.
 Dans cette étude, les individus sont échantillonnés sur des distances parfois trop importantes (\textit{P. pedemontana}) ou trop courtes (\textit{P. pedemontana} des Écrins ou alors \textit{P. apennina}).



\newpage
\input{tex/thx.tex}

%%%%%%%%%%%%%%%%%%%%%%%%%%%%%%%%%%%%%%%%%%%%%%%%%%%%%%%%%%%%%%%%%%%%%%
% Références
%%%%%%%%%%%%%%%%%%%%%%%%%%%%%%%%%%%%%%%%%%%%%%%%%%%%%%%%%%%%%%%%%%%%%%
%\newpage
\section{Bibliographie}
\bibliographystyle{authordate1}
\bibliography{Primula}

%%%%%%%%%%%%%%%%%%%%%%%%%%%%%%%%%%%%%%%%%%%%%%%%%%%%%%%%%%%%%%%%%%%%%%
% Ressources
%%%%%%%%%%%%%%%%%%%%%%%%%%%%%%%%%%%%%%%%%%%%%%%%%%%%%%%%%%%%%%%%%%%%%%
\section{Ressources}

\begin{web}[h]
	\caption{\url{http://www.ecrins-parcnational.fr/actualite/lenigme-de-la-primevere-du-valgaudemar}}
	\label{pne}
\end{web}

\begin{web}[h]
	\caption{\url{https://github.com/gowachin/P._pedemontana}}
	\label{github}
\end{web}



%%%%%%%%%%%%%%%%%%%%%%%%%%%%%%%%%%%%%%%%%%%%%%%%%%%%%%%%%%%%%%%%%%%%%%
% Annexes
%%%%%%%%%%%%%%%%%%%%%%%%%%%%%%%%%%%%%%%%%%%%%%%%%%%%%%%%%%%%%%%%%%%%%%
\newpage
\begin{landscape}

\begin{annexe}
	\centering
\rowcolors{2}{white}{gray!25}
\resizebox{27cm}{!}{%
	\begin{tabular}{cccccccccccc}
	\toprule
Species &Locality &Code &Morph &Collector &Date &Longitude &Latitude &Altitude&Reads raw &Reads trimmed &Voucher \\
	\midrule
P. apennina* &Sella del Marmagna, Italy &AMB &Short-styled &F. Boucher/L. Gallien &30/05/14 & 10.00575 & 44.3978 &1610&6885928&6486849&Photo \\
P. apennina &Monte Marmagna, Italy &AML &Long-styled &F. Boucher/L. Gallien &30/05/14 & 9.99731 & 44.39672 &1825&1856867&1663377&Photo \\
P. apennina &Monte Orsaro, Italy &AOL &Long-styled &F. Boucher/L. Gallien &30/05/14 & 9.99666 & 44.39883 &1818&3494081&3230296&Photo \\
P. cottia &Below locus classicus, Italy &CS1 & NA &F. Boucher &23/07/14 & 7.0716 & 44.9271 &1159&5127416&4814386&Photo \\
P. cottia &Prali, locus classicus, Italy &CP1 & NA &F. Boucher &23/07/14 & 7.06583 & 44.9186 &1407&3160322&2941542&Photo \\
P. cottia &Prali, locus classicus, Italy &CP4 & NA &F. Boucher &23/07/14 & 7.06583 & 44.9186 &1407&3482252&3201012&Photo \\
P. daonensis &Passo di Gavia, Italy &DGB &Short-styled &F. Boucher/L. Gallien &27/05/14 & 10.49701 & 46.31843 &2219&6095146&5757485&Photo \\
P. daonensis &Ritorto, Italy &DRL &Long-styled &F. Boucher/L. Gallien &27/05/14 & 10.80429 & 46.23149 &2083&4607717&4299840&Photo \\
P. hirsuta &Malga Bordolona, Italy &DMB &Short-styled &F. Boucher &09/06/14 & 10.87383 & 46.43412 &2214&6073360&5722516&Photo \\
P. hirsuta &Refuge du Couvercle, France &HC1 & NA &C. Dentant &15/07/14 & 6.9656 & 45.9103 &2649&2639620&2384576& NA \\
P. hirsuta &Grand Chat, France &HGL &Long-styled &F. Boucher/L. Gallien &18/05/14 & 6.2147 & 45.4467 &1986&4583323&4270118&Photo \\
P. hirsuta &Steibensee, Switzerland &HS2 & NA &F. Boucher &07/09/14 & 8.17 & 46.45 &2414&2891228&2643122&Photo \\
P. hirsuta &Pic du Midi d'Ossau, France &HP1 & NA &C. Roquet &15/08/14 & -0.4381 & 42.8431 &2739&1881282&1772502& NA \\
P. hirsuta &Passo del Bernina, Switzerland &HPB &Short-styled &F. Boucher &09/06/14 & 10.02717 & 46.41069 &2328&6566463&6195227&Photo \\
P. pedemontana &Barrage de Tignes, France &PT1 & NA &F. Boucher &27/07/14 & 6.94633 & 45.4805 &1836&6515454&6086010&YES \\
P. pedemontana &Vallon d'Avérole, France &PV1 & NA &F. Boucher &27/07/14 & 7.08707 & 45.29356 &2144&6480484&6100355&YES \\
P. sp. Lauzon Valley &Lauzon Valley, France &GA2 & NA &P. Salomé/R. Bonet/F. Boucher &25/07/14 & 6.2784 & 44.8418 &1732&4150458&3873470&YES \\
P. sp. Lauzon Valley &Lauzon Valley, France &GA4 & NA &P. Salomé/R. Bonet/F. Boucher &25/07/14 & 6.2773 & 44.8366 &1899&4796528&4489119&YES \\
P. villosa ssp. irmingardis &Rappolt Kogel, Austria &VR3 &Short-styled &F. Boucher &07/06/14 & 14.88541 & 47.08313 &1871&4722814&4234314&Photo \\
P. villosa ssp. irmingardis &Rappolt Kogel, Austria &VR1 &Long-styled &F. Boucher &07/06/14 & 14.88541 & 47.08313 &1871&4789459&4283316&Photo \\
P. villosa ssp. villosa &Turracher Hohe, Austria &VL2 &Long-styled &F. Boucher &07/06/14 & 13.87581 & 46.91273 &1801&3227112&2776420&Photo \\
P. villosa ssp. villosa &Turracher Hohe, Austria &VB1 &Short-styled &F. Boucher &07/06/14 & 13.87581 & 46.91273 &1801&3004397&2708593&Photo \\
	\bottomrule
	\end{tabular}}
	\caption{\textbf{Individus séquencés pour cette étude}. D'après les informations de \citet{Boucher2016a}.}
	\label{table_ind}
\end{annexe}
\end{landscape}

\end{document}

%%%%%%%%%%%%%%%%%%%%%%%%%%%%%%%%%%%%%%%%%%%%%%%%%%%%%%%%%%%%%%%%%%%%%%
% Figures et raccoucis!
%%%%%%%%%%%%%%%%%%%%%%%%%%%%%%%%%%%%%%%%%%%%%%%%%%%%%%%%%%%%%%%%%%%%%%

%\begin{figure}[!ht]
%    \centering
%    \includegraphics[width=0.9\textwidth]{ACP.png}
%    \caption{\textbf{Analyse en composante principale des variables biochimiques mesurées sur le sol des deux sites étudiés.} La représentation de l'axe 1/axe 2 permet d'observer 76.94\% de l'inertie totale comme représenté sur l'éboulis de valeur propre en haut à droite.}
%    \label{ACP}
%    \centering
%\end{figure}

%\begin{wrapfigure}{t}{0.5\textwidth}
%\lipsum[1]
%\end{wrapfigure}

%\begin{wrapfigure}{r}{0.50\textwidth}
%	\vspace{-30pt}
%	\begin{center}
%		\includegraphics[width=0.49\textwidth]{ACP.png}
%	\end{center}
%	\vspace{-20pt}
%	\caption{\textbf{Analyse en composante principale des variables biochimiques mesurées sur le sol des deux sites étudiés.} La représentation de l'axe 1/axe 2 permet d'observer 76.94\% de l'inertie totale comme représenté sur l'éboulis de valeur propre en haut à droite.}
%	\label{ACPw}
%\end{wrapfigure}

%\begin{figure}[h]
%\begin{subfigure}{0.5\textwidth}
%\includegraphics[width=0.9\linewidth, height=5cm]{lion-logo}
%\caption{Caption1}
%\label{fig:subim1}
%\end{subfigure}
%\begin{subfigure}{0.5\textwidth}
%\includegraphics[width=0.9\linewidth, height=5cm]{mesh}
%\caption{Caption 2}
%\label{fig:subim2}
%\end{subfigure}
%\caption{Caption for this figure with two images}
%\label{fig:image2}
%\end{figure}

%\begin{table}[!ht]\footnotesize
%	\centering
%	\begin{tabular}{ccccccccc}
%	\toprule
%	\multicolumn{1}{c}{} & \multicolumn{2}{c}{Prairie} & \multicolumn{2}{c}{Forêt} & \multicolumn{4}{c}{Test Statistique} \\
%	\midrule
%	\multicolumn{1}{c}{Mesures} & {Valeur} & {Écart-type}  & {Valeur} & {Écart-type} & {Test} & {P-valeur} & {Df} & {Statistique} \\
%	\midrule
%	pH & 7.37 & 0.46 & 5.52 & 0.71 & Student-T & \(5.18.10^{-7}\) & 20 & -7.25\\

%	Teneur en eau & 20.63 & 7.26 & 32.93 & 12.57 & Mann-Whitney & \(8.11.10^{-4}\) & - & 112\\

%	Matière Organique & \textbf{6.63} & 0.89 & \textbf{9.19} & 1.87 & Welch-T & \(10^{-3}\) & 14.32 & 4.09\\

%	TDN & \textbf{32.60} & 6.48 & \textbf{64.63} & 19.39 & Welch-T & \(2.1.10^{-4}\) & 12.21 & 5.20\\

%	DON & \textbf{25.52} & 5.07 & \textbf{53.58} & 14.02 & Welch-T & \(3.48.10^{-5}\) & 12.57 & 6.24\\

%	Mineralisation & 3.42 & 1.54 & 5.59 & 2.39 & Student-T & \(2.10^{-2}\) & 20 & 2.53\\

 %   FDA & 4.35 & 1.41 & 6.86 & 1.62 & Student-T & \(10^{-4}\) & 18 & 3.70\\
%	\bottomrule
%	\end{tabular}
%	\caption{\textbf{Mesures biochimiques des deux sites d'échantillonnages} Chaque échantillon est constitué de 11 mesures, à l'exception de la mesure de FDA avec 10 mesures pour cause de mesures aberrantes, Tout les test présentent une p-valeur significative pour un risque \textit{alpha} de 5\%. Les valeurs en gras sont des moyennes, tantis que les autres sont les médianes.}
%	\label{table_site}
%\end{table}

%\begin{annexe}
%\centering
%\includegraphics[height=\textwidth , angle=270]{dessin_collembole.png}
%\caption{Dessin d'observation de Collembole}
%\label{collembole}
%\centering
%\end{annexe}
