\documentclass[12pt,a4paper,notitlepage]{article}
%%%%%%%%%%%%%%%%%%%%%%%%%%%%%%%%%%%%%%%%%%%%%%%%%%%%%%%%%%%%%%%%%%%%%
% Template pour rendus Master
%%%%%%%%%%%%%%%%%%%%%%%%%%%%%%%%%%%%%%%%%%%%%%%%%%%%%%%%%%%%%%%%%%%%%%
%\documentclass[12pt,a4paper,titlepage]{report}
\usepackage[utf8]{inputenc} %encodage
\usepackage[square,sort,comma,numbers]{natbib}
\setcitestyle{authoryear}
\renewcommand{\bibsection}{}

\usepackage[french]{babel}
%mise en page générale
\usepackage{geometry} 
%\geometry{a4paper} % format de feuille
\geometry{top=2.5cm, bottom=2.5cm, left=2.5cm, right=2.5cm} %marges
%Police
\usepackage{mathptmx} % Times si compilateur pdfLatex
%interligne
\linespread{1.5}

%en tete et pied de page
\usepackage{fancyhdr}
\usepackage{lastpage}
\pagestyle{plain}


\usepackage{url, lipsum} %Lorem ipsum

\usepackage{wrapfig} %position d'images dans le texte
%\usepackage[myheadings]{fullpage}
\usepackage{graphicx, subcaption, setspace, booktabs}

\usepackage{caption}
\DeclareCaptionType{annexe}[Annexe][Liste d'annexes]

\usepackage{lipsum}

\usepackage{hyperref}
\usepackage{subcaption}

\usepackage[export]{adjustbox}

\DeclareRobustCommand{\rchi}{{\mathpalette\irchi\relax}}
\newcommand{\irchi}[2]{\raisebox{\depth}{$#1\chi$}}

%%%%%%%%%%%%%%%%%%%%%%%%%%%%%%%%%%%%%%%%%%%%%%%%%%%%%%%%%%%%%%%%%%%%%%
% Page de garde et table des matières
%%%%%%%%%%%%%%%%%%%%%%%%%%%%%%%%%%%%%%%%%%%%%%%%%%%%%%%%%%%%%%%%%%%%%%
\title{TITRE}
\author{Maxime Jaunatre, Master 1 BEE Grenoble}
\date{Mai 2019}


\renewcommand*\contentsname{Table des matières}

\begin{document}

\begin{titlepage} %Page de garde!!

\begin{figure}
\centering
\begin{subfigure}{.5\textwidth}
  \centering
  \includegraphics[width=.8\linewidth]{fig/UGA.jpg}
  \label{fig:sub1}
\end{subfigure}%
\begin{subfigure}{.5\textwidth}
  \centering
  \includegraphics[width=.8\linewidth]{fig/leca.jpg}
  \label{fig:sub2}
\end{subfigure}
\label{fig:test}
\end{figure}

\maketitle

\begin{figure}[h]
\begin{center}
\includegraphics[scale=3]{fig/primulapedemontana_7.jpg}
\end{center}
\end{figure}

%\begin{wrapfigure}{r}{0.50\textwidth}
%	\vspace{-30pt}
%	\begin{center}
%		\includegraphics[width=0.49\textwidth]{ACP.png}
%	\end{center}
%	\vspace{-20pt}
%	\caption{\textbf{Analyse en composante principale des variables biochimiques mesurées sur le sol des deux sites étudiés.} La représentation de l'axe 1/axe 2 permet d'observer 76.94\% de l'inertie totale comme représenté sur l'éboulis de valeur propre en haut à droite.}
%	\label{ACPw}
%\end{wrapfigure}

\newpage
\begin{abstract} %abstract!!!
\begin{center}

\lipsum[1]

\end{center}
\end{abstract}

\end{titlepage}

\tableofcontents
\newpage


%%%%%%%%%%%%%%%%%%%%%%%%%%%%%%%%%%%%%%%%%%%%%%%%%%%%%%%%%%%%%%%%%%%%%%
% BODY
%%%%%%%%%%%%%%%%%%%%%%%%%%%%%%%%%%%%%%%%%%%%%%%%%%%%%%%%%%%%%%%%%%%%%%

%Consignes pour rapport et soutenance du stage M1

%Le rapport
%Il doit comporter 12 pages (+/1page) figures comprises, sommaire et références biblio en plus, annexes déconseillées. Moins de 20 références bibliographiques, 12 lignes de résumé au dos de la couverture.
%Le plan à adopter a priori est « Introduction / Matériel et méthodes / Résultats / Discussion » à moins qu’il ne soit pas adapté à votre travail (c’est rarement le cas).
%Format : TIMES 12 points, avec un interligne 1,5. Le rapport peut être rédigé en Français ou en Anglais.
%Les personnes qui font un stage long et concentrent leur rapport sur la mise au point de la méthodologie (c'est à dire n'ayant pas de résultats au moment de la soutenance) sont encouragées à expliquer ce contexte dans un court préambule d'une dizaine de lignes maximum. Les membres de jury sont informés de l'éventualité d'une telle situation. 
%Le rapport de stage est à rendre à votre responsable pédagogique de stage pour le :
%Mardi 28 mai 2018 à 12 h au plus tard
%Une version papier adressée par courrier ou déposée au secrétariat du LECA. Une version pdf envoyée par mail au tuteur de stage et au responsable du M1.
%L’envoi uniquement au format électronique peut se faire à titre exceptionnel (stage à l’étranger nécessitant un long délai d’acheminement – contacter par email votre responsable pédagogique avant).

%La soutenance
%Modalités : 10 min de présentation + 5 min de questions devant un jury de 3 personnes. Le maître de stage est invité à venir écouter la soutenance s’il le souhaite. 
%Format de présentation : fichier pdf ou ppt sur clé USB. Les soutenances auront lieu la semaine du 3 Juin 2019. Un planning de soutenance vous sera communiqué fin mai. Les étudiants effectuant un stage long à l’étranger et ne rentrant pas en France peuvent soutenir par visioconférence.
%Contacter votre responsable pédagogique par email mi mai pour fixer les modalités.

%Fiche d’évaluation
%La fiche d'évaluation doit être donnée à votre encadrant en début de stage. Elle sera renvoyée par email par le maître de stage au plus tard le 1 er juin 2019.
%Attention : pour les stages longs à l'étranger prévoyez l'éventualité d'un retour à Grenoble en cas de rattrapage (seconde session d'examen deuxième quinzaine de juin première quinzaine de juillet).

\section{Introduction}

\lipsum[1-2]


%
\section{Matériels et méthodes}

\subsection{Échantillonnage}

Les données génétiques utilisés lors de cette étude ont été produite dans le cadre de l'étude précédente de \citet{Boucher2016a}. Il s'agit d'un jeu de donnée composé de 90 individus des espèces composant la section \textit{Auricula}, au sein du genre \textit{Primula}. Ces individus ont été prélevés à travers les Alpes entre avril et septembre 2014. L'identification taxonomique des individus a été réalisé sur le terrain, mais un individu a du être réattribué après analyse génétique au taxon \textit{P. hirsuta}.

Les SNPs sont issus de séquençage haut débit via hyRAD \citep{Suchan2016}. Cette technique permet de génotyper le long du génome malgré des mutations sur les sites de restrictions. En effet les enzymes de restrictions sont trop sensibles à la mutation d'un nucléotide, tandis que les sondes ARN peuvent s'hybrider sur des sites plus nombreux sur le génome. La nécessité de capturer des sites malgré une faible variation provient du niveau interspécifique de l'étude, qui pose l'hypothèse que les mutations peuvent se placer sur les sites de restrictions et ainsi limiter leur capture par simple séquençage ddRAD.

%We employ a newly developed next-generation-sequencing protocol, involving sequence capture with RAD probes, and map reads to the reference genome of Primula veris to obtain DNA matrices with thousands of SNPs

%Total DNA was extracted from silica-dried leaves using a DNeasy Plant Mini kit (Qiagen, Hilden, Germany) following the manufacturer’s instructions. DNA quality was visualized on 0.8% agarose gels and quantity was assessed using a QuBit 2.0 fluorom- eter (2.0, Life Technologies, Carlsbad, CA, USA). Genomic DNA was converted into RAD-Capture genotyping-by-sequencing libraries (SNPsaurus, LLC), a new protocol aimed at harnessing the wide genomic spectrum of RAD-sequencing while reducing the amount of missing data in interspecific datasets. Briefly, a double-digest RAD library was created from 100 ng of a pool of genomic DNA containing a diverse set of individuals from sect. Auricula (belong- ing to P. allionii, P. apennina, P. auricula, P. glutinosa, and P. minima; see Table A.1). The pooled DNA was digested with PstI-HF and MfeI- HF (NEB) and ligated to complementary adapters that allowed the resulting amplified fragments to be converted to biotinylated RNA baits. Fragments with inserts roughly 100–350 bp in size were iso- lated by gel extraction from a portion of the ligated product prior to amplification and the in vitro transcription reaction to create the RNA baits. Shotgun sequencing libraries were prepared from the 90 study samples using 5 ng each in a 1/10th Nextera (Illumina, Inc) reaction with unique dual-indexes to distinguish the individu- als. The samples were pooled and size-selected for insert sizes roughly 170–370 bp. The pooled libraries were then used in two successive overnight hybridizations to the biotinylated bait library, followed by capture using DynabeadsÒ MyOneTM Streptavidin C1 magnetic beads (Thermo Fisher) and amplification. The final cap- tured products were sequenced in a single 150 bp NextSeq 500 High Output run at the Genomics and Cell Characterization Core of the University of Oregon.

\subsection{Bioinformatique}

Pour chacun des individus, l'information consiste en une séquence de SNPs appellés par Freebayes, exporté sous format VCF. Les analyses ont été porté sur deux jeux de données différents car filtrés sous des seuils différents.
Le premier jeu de donnée ('\verb|m30_-q20_mincov20|') est issus de filtres très strict, avec un score de qualité (Phred) requis de 30 et une couverture minimale de 20 par site. Afin de ne pas biaiser l'analyse par des seuils favorisant les régions conservées, le second jeu de données ('\verb|m13_-q20_mincov10|') est quant à lui produit avec un Phred minimal de 13 et une couverture de 10. A partir de ces séquences, les SNPs ont été isolés par Freebayes, avec un score de Phred minimal de 20 et un support de lecture de 30\% minimum par allèle.

%parler du pipeline
A partir des deux jeux de donnée initiaux, un pipeline est établis pour générer divers ensembles de données. Cette automatisation a permit entre autre de pouvoir évaluer l'effet des seuils posés au fur et à mesure de l'analyse. Les fonctions sont rassemblées en un package R hébergé sur Github (lien web \ref{github})
Dans un premier temps, le fichier initial est traité sous Rstudio \citep{RTeam2017}, avec la fonction \verb|subset_reorder|, qui permet de reconstruire le fichier en ne gardant que les individus souhaité dans l'ordre indiqué. La fonction suivante \verb|rare|, permet de trier les allèles considérés comme étant présents dans un trop faible pourcentage des individus. Ces allèles rares sont écartés du jeu de donnée et le loci pour l'individu présentant cet allèle rare est considéré comme une donnée manquante. Cette étape permet également de supprimer les lectures avec de multiples allèles variants qui sont reconnus comme des artefacts des algorithmes utilisés pour appeler les SNP.  %We then filtered variants in a very conservative way by removing all called multiple nucleotide polymorphisms, indels and complex variants. These types of variants are known to often be artifacts generated by the algorithms used for short read mapping (see https://www. broadinstitute.org/gatk/guide/best-practices.php).
Suite aux deux tris précédents, il y a donc des loci pour lesquels tout les individus portent la même information. Afin de ne garder que les loci polymorphiques, une fonction \verb|clean| est donc appelée à la fin de la fonction \verb|rare|, pour supprimer les locis monomorphiques.
Les loci sont aussi filtré sur le score de qualité (\verb|QUAL|) indiqué dans le vcf, qui correspond a la confiance dans l'assignation de l'allèle variant. %In the context of variant calling, Phred-scaled quality scores can be used to represent many types of probabilities. The most commonly used in GATK is the QUAL score, or variant quality score. It is used in much the same way as the base quality score: the variant quality score is a Phred-scaled estimate of how confident we are that the variant caller correctly identified that a given genome position displays variation in at least one sample
Une fonction de \verb|tri| est appelé pour supprimer les loci puis individus présentant trop de données manquantes selon les seuils posés.
Enfin, afin de limiter le déséquilibre de liaisons, les sites sont filtrés selon leurs positions sur les contigs, où une distance minimale \verb|n| doit être prise en compte entre deux sites d'un même contig pour que le second site soit conservé.


Afin de pouvoir utiliser les fichiers triés sous divers format, la sous-sélection d'individus et de loci est enregistrée sous quatre formats : \verb|.vcf|, \verb|.str|, \verb|.geno|, \verb|.snp|. La transformation d'un format en un autre se fait respectivement au moyen d'un script bash, par l'utilisation du software PGDSpider \citep{Lischer2012}, le package LEA \citep{Frichot2015} et un script R. La production de ces fichiers est inscrite dans le pipeline par la fonction \verb|files|.

L'ensemble de ces fonctions sont indépendantes mais peuvent être appellées dans le bon ordre au moyen de la fonction \verb|dataset|, qui prend en entrée un vecteur de noms d'individus, un \verb|csv| contenant les assignations aux populations, le fichier \verb|vcf| original, le nom des fichiers de sortie et les seuils précisés au-dessus.


Chaque seuil est choisis après vérification que la décision n'impacte que peu les résultats de diversité génétique.
Le nombre d'individus étant faible, il faut optimiser le nombre de SNP par individus car cela permet de limiter la perte d'information et d'atteindre les mêmes résultats qu'attendus avec un échantillon plus vaste. \citep{Nazareno2017}


\subsection{Génétique des populations}

Les analyses de génétique des populations ont été réalisés sous Rstudio \citep{RTeam2017} avec différents packages. Le package adegenet a été utilisé pour inférer les fst, le package LEA pour étudier la structure de la population, package Introgress pour mesurer les introgressions entre P. pedemontana et P. hirsuta.

admixture aussi a peut etre été utilisé pour ça.

\subsection{Inférences bayesiennes}

Afin de caractériser l'admixture probable entre le taxon des Écrins et \textit{P. hirsuta}, une approche par approximate Bayesian computation (ABC) a été réalisée sur le logiciel DIYABC \cite{Cornuet2014}. Ce logiciel permet de simuler des jeux de données selon divers scénarios, en échantillonnant des paramètres entre des priors définis. Les scénarios sont ensuite classés selon les probabilités a posteriori d'observer notre jeu de donnée initial selon les scénarios proposés. Seuls quelques scénarios ont été étudiés ici, en prenant en compte le fait qu'ils sont à chaque fois supportés par peu d'individus. Les priors sont également proposés dans un grand interval et selon une distribution uniforme, les temps de divergence entre populations et taille de population n'ayant pas été étudiés sur le terrain.
%with the approximate Bayesian computation (ABC) approach implemented in the software DIYABC (Cornuet et al. 2014). In short, a large number of simu- lated data sets were produced under each tested scenario by sampling parameter values into predefined prior dis-tributions, and an estimation of each scenario posterior probability was calculated as a function of the similarity between several summary statistics of simulated data sets and observed data using a logistic regression proce-dure. These posterior probabilities enable a ranking of scenarios and allow for the identification of the most probable evolutionary history. We only considered five evolutionary scenarios (Fig. 2), based on previously pub-lished studies dealing with the evolution of the complex (Porter et al. 1994; Wiemers 1998; Schmitt & Besold 2010) and our own hypothesis based on the morphometric and phylogenetic analyses (see Results). Scenario design and simulations. A uniform distribution with a large interval was chosen for each prior increas-ing the need in computation time but overcoming the lack of knowledge on population sizes, divergence times and admixture rate (see Table S2 Supporting information). Only one approximate C. arcania /C. gardetta divergence time of 1.5–4 million of years was available from Kodandaramaiah & Wahlberg (2009). For each scenario, a total of one million data sets were sim-ulated and the posterior probability was computed by performing a logistic regression on the 1% of simulated data closest to the observed data set (Cornuet et al.2008, 2010). Summary statistics used for observed/sim- ulated data sets comparisons are the mean gene diver-sity and Fst across all loci and Nei’s distances among populations. Gene diversity was chosen because allelic richness is directly influenced by hybridization; the genetic differentiation (Fst) and distance (Nei’s) among taxa were used because these indexes reflect the degree of differentiation between populations and are highly relevant to compare evolutionary histories. Assessing the ‘goodness of fit’ of the model. Once the most probable scenario was identified, different posterior anal-yses were carried out to evaluate the trustfulness of the procedure. First, to be sure of the choice of the scenario, 500 new data sets were simulated with each historical model creating pseudo-observed data sets for which pos-terior probabilities were re-estimated with the same pro-cedure described above. Type I and type II errors were evaluated by measuring, respectively, the fraction of data sets simulated under the best scenario that were assigned to other scenarios and the fraction of data sets simulated under other scenarios that were assigned to the best sce- nario. Second, we wanted to know whether our selected model is able to produce data sets similar to the observed one. This confidence in the model is evaluated by esti-mating the similarity between simulated (1000 simula-tions) and real data sets using summary statistics different than the ones used for model choice (an impor-tant precaution, see Cornuet et al. 2010). For each sum-mary statistic, a P-value is estimated by ranking the observed value among the values obtained with simu-lated data sets, and a principal component analysis was also performed to check visually the position of the observed data in relation to the data sets generated from the posterior predictive distribution. Posterior distribution of parameters and their precision. It was possible to evaluate the posterior distributions of parameters for our selected scenario using a local linear regression on the 1% of simulations closest to the observed data set (Cornuet et al. 2010). These distribu-tions gave an idea of the most probable value (median) and the ‘approximation’ (width of the distribution) for each historical model. A precision of these estimations was evaluated by computing the median of the absolute error divided by the true parameter value of the 500 pseudo-observed data sets simulated under the selected scenario using the median of the posterior distribution as point estimate (Cornuet et al. 2010).

\section{Matériels et méthodes}

\subsection{Échantillonnage}

Les données génétiques utilisés lors de cette étude ont été produite dans le cadre de l'étude précédente de \citet{Boucher2016a}. L'identification taxonomique des individus a été réalisé sur le terrain.

Les SNPs sont issus de séquences obtenus par séquençage via sondes RAD. Cette technique permet de génotyper le long du génome malgré des mutations sur les sites de restrictions. En effet les enzymes de restrictions sont trop sensibles à la mutation d'un nucléotide, tandis que les sondes ARN peuvent s'hybrider sur des sites plus nombreux sur le génome. La nécessité de capturer des sites malgré une faible variation provient du niveau interspécifique de l'étude, qui pose l'hypothèse que les mutations peuvent se placer sur les sites de restrictions et ainsi limiter leur capture par séquençage ddRAD.

%We employ a newly developed next-generation-sequencing protocol, involving sequence capture with RAD probes, and map reads to the reference genome of Primula veris to obtain DNA matrices with thousands of SNPs

%Total DNA was extracted from silica-dried leaves using a DNeasy Plant Mini kit (Qiagen, Hilden, Germany) following the manufacturer’s instructions. DNA quality was visualized on 0.8% agarose gels and quantity was assessed using a QuBit 2.0 fluorom- eter (2.0, Life Technologies, Carlsbad, CA, USA). Genomic DNA was converted into RAD-Capture genotyping-by-sequencing libraries (SNPsaurus, LLC), a new protocol aimed at harnessing the wide genomic spectrum of RAD-sequencing while reducing the amount of missing data in interspecific datasets. Briefly, a double-digest RAD library was created from 100 ng of a pool of genomic DNA containing a diverse set of individuals from sect. Auricula (belong- ing to P. allionii, P. apennina, P. auricula, P. glutinosa, and P. minima; see Table A.1). The pooled DNA was digested with PstI-HF and MfeI- HF (NEB) and ligated to complementary adapters that allowed the resulting amplified fragments to be converted to biotinylated RNA baits. Fragments with inserts roughly 100–350 bp in size were iso- lated by gel extraction from a portion of the ligated product prior to amplification and the in vitro transcription reaction to create the RNA baits. Shotgun sequencing libraries were prepared from the 90 study samples using 5 ng each in a 1/10th Nextera (Illumina, Inc) reaction with unique dual-indexes to distinguish the individu- als. The samples were pooled and size-selected for insert sizes roughly 170–370 bp. The pooled libraries were then used in two successive overnight hybridizations to the biotinylated bait library, followed by capture using DynabeadsÒ MyOneTM Streptavidin C1 magnetic beads (Thermo Fisher) and amplification. The final cap- tured products were sequenced in a single 150 bp NextSeq 500 High Output run at the Genomics and Cell Characterization Core of the University of Oregon.

\subsection{Bioinformatique}

Le jeu de données initial contient 90 individus pour les espèces échantillonnées initialement au sein de la section Auricula. Pour chacun de ces individus, l'information consiste en une séquence de SNPs appellés par Freebayes. Les analyses ont été porté sur deux jeux de données différents car extrait de Freebayes sous des seuils différents.

%parler du pipeline
A partir des deux jeux de donnée initial, un pipeline est établis pour générer divers ensembles de données. Cette automatisation a permit entre autre de pouvoir évaluer l'effet des seuils posés au fur et à mesure de l'analyse. Les fonctions sont rassemblées en un package R hébergé sur Github (\autoref{app:github})
Dans un premier temps, le fichier initial est traité sous Rstudio \citep{RTeam2017}, avec la fonction subset$\_$reorder, qui permet de reconstruire le fichier en ne gardant que les individus souhaité dans l'ordre indiqué. La fonction suivante, rare, permet de trier les alleles considérés comme étant présents dans un trop faible pourcentage des individus. Ces allèles rares sont écartés du jeu de donnée et le loci pour l'individu présentant ce loci sont considéré comme une donnée manquante. Suite aux deux tris précédents, il y a donc des loci pour lesquels tout les individus portent la même information. Afin de ne garder que les loci polymorphiques, une fonction clean est donc appelée à la fin de la fonction rare, afin de supprimer les locis monomorphiques.
Enfin une dernière fonction de tri est appelé pour supprimer les loci puis individus présentant trop de données manquantes. 

Afin de pouvoir utiliser les fichiers triés sous divers format, la sous-selection d'individus et de loci est enregistrée sous format vcf, au moyen d'une fonction appelant un script bash.


Chaque seuil est choisis après vérification que la décision n'impacte que peu les résultats de diversité génétique.
Le nombre d'individus étant faible, il faut optimiser le nombre de SNP par individus car cela permet de limiter la perte d'information et d'atteindre les mêmes résultats qu'attendus avec un échantillon plus vaste. \citep{Nazareno2017}

\subsection{Génétique des populations}

Les analyses de génétique des populations ont été réalisés sous Rstudio.\citep{RTeam2017}

\section{Résultats}

\subsection{Tri du jeu de données}

\lipsum[1-2]

%\begin{wrapfigure}{r}{0.50\textwidth}
%	\vspace{-30pt}
%	\begin{center}
%		\includegraphics[width=0.49\textwidth]{ACP.png}
%	\end{center}
%	\vspace{-20pt}
%	\caption{\textbf{Analyse en composante principale des variables biochimiques mesurées sur le sol des deux sites étudiés.} La représentation de l'axe 1/axe 2 permet d'observer 76.94\% de l'inertie totale comme représenté sur l'éboulis de valeur propre en haut à droite.}
%	\label{ACPw}
%\end{wrapfigure}


%\begin{table}[!ht]\footnotesize
%	\centering
%	\begin{tabular}{ccccccccc}
%	\toprule
%	\multicolumn{1}{c}{} & \multicolumn{2}{c}{Prairie} & \multicolumn{2}{c}{Forêt} & \multicolumn{4}{c}{Test Statistique} \\
%	\midrule
%	\multicolumn{1}{c}{Mesures} & {Valeur} & {Écart-type}  & {Valeur} & {Écart-type} & {Test} & {P-valeur} & {Df} & {Statistique} \\
%	\midrule
%	pH & 7.37 & 0.46 & 5.52 & 0.71 & Student-T & \(5.18.10^{-7}\) & 20 & -7.25\\
	
%	Teneur en eau & 20.63 & 7.26 & 32.93 & 12.57 & Mann-Whitney & \(8.11.10^{-4}\) & - & 112\\
	
%	Matière Organique & \textbf{6.63} & 0.89 & \textbf{9.19} & 1.87 & Welch-T & \(10^{-3}\) & 14.32 & 4.09\\
	
%	TDN & \textbf{32.60} & 6.48 & \textbf{64.63} & 19.39 & Welch-T & \(2.1.10^{-4}\) & 12.21 & 5.20\\
	
%	DON & \textbf{25.52} & 5.07 & \textbf{53.58} & 14.02 & Welch-T & \(3.48.10^{-5}\) & 12.57 & 6.24\\
	
%	Mineralisation & 3.42 & 1.54 & 5.59 & 2.39 & Student-T & \(2.10^{-2}\) & 20 & 2.53\\

 %   FDA & 4.35 & 1.41 & 6.86 & 1.62 & Student-T & \(10^{-4}\) & 18 & 3.70\\
%	\bottomrule
%	\end{tabular}
%	\caption{\textbf{Mesures biochimiques des deux sites d'échantillonnages} Chaque échantillon est constitué de 11 mesures, à l'exception de la mesure de FDA avec 10 mesures pour cause de mesures aberrantes, Tout les test présentent une p-valeur significative pour un risque \textit{alpha} de 5\%. Les valeurs en gras sont des moyennes, tantis que les autres sont les médianes.}
%	\label{table_site}
%\end{table}

%\begin{figure}[!ht]
%    \centering
%    \includegraphics[width=0.9\textwidth]{ACP.png}
%    \caption{\textbf{Analyse en composante principale des variables biochimiques mesurées sur le sol des deux sites étudiés.} La représentation de l'axe 1/axe 2 permet d'observer 76.94\% de l'inertie totale comme représenté sur l'éboulis de valeur propre en haut à droite.}
%    \label{ACP}
%    \centering
%\end{figure}


\subsection{Clade Hirsuta}

\lipsum[1]

\subsection{Complexe ouest alpin}

\lipsum[1]

\section{Discussion}

\lipsum[1-4]

\paragraph{Remerciements}
Je tiens enfin à remercier sincèrement les personnes qui m'ont aidé durant ce stage.
Tout d'abord Florian Boucher qui, en qualité de maître de stage, m'a accueilli au sein du LECA parmi l'équipe \textbf{EQUIPE???} et confié ses données.
Ce stage m'a permis de me confronter à un sujet d'étude passionnant et je tiens donc à lui témoigner ma plus grande reconnaissance, en particulier pour sa patience, sa disponibilité et ses explications. Je remercie également Stéphanie Sherpa qui m'aura guidé durant mes recherches en génétique des populations.

Je n’oublie pas les autres membres du LECA, qui auront contribué à rendre mon stage agréable par leur attitude chaleureuse et bienveillante à mon égard.
Je remercie enfin Caroline Kebaili, \textbf{MEH?} et \textbf{MEH?}, stagiaires de M2, avec qui j’ai partagé la salle 308 durant ces deux mois et qui ont toujours répondu présent à la moindre de mes questions.

%%%%%%%%%%%%%%%%%%%%%%%%%%%%%%%%%%%%%%%%%%%%%%%%%%%%%%%%%%%%%%%%%%%%%%
% Références
%%%%%%%%%%%%%%%%%%%%%%%%%%%%%%%%%%%%%%%%%%%%%%%%%%%%%%%%%%%%%%%%%%%%%%
\newpage
\section{Bibliographie}
\bibliographystyle{authordate1}
\bibliography{Primula}

\section{Ressources}
\url{https://github.com/gowachin/Pedemontana}\label{app:github}

%\newpage
%\begin{annexe}
%\centering
%\includegraphics[height=\textwidth , angle=270]{dessin_acarien.png}
%\caption{Dessin d'observation d'Acarien}
%\label{acarien}
%\centering
%\end{annexe}

%\begin{annexe}
%\centering
%\includegraphics[height=\textwidth , angle=270]{dessin_collembole.png}
%\caption{Dessin d'observation de Collembole}
%\label{collembole}
%\centering
%\end{annexe}

\end{document}



% SNIPPETS
%-------------------------------------------------------------------------------
%\begin{figure}[!ht]
%	\centering
%	\includegraphics[width=0.8\textwidth]{graph}
%	\caption{Blood pressure ranges and associated level of hypertension (American Heart Association, 2013).}
%	\centering
%	\label{label:graph}
%\end{figure}
%
%\begin{wrapfigure}{t}{0.5\textwidth}
%\lipsum[10]
%\end{wrapfigure}
%
%\begin{wrapfigure}{r}{0.30\textwidth}
%	\vspace{-40pt}
%	\begin{center}
%		\includegraphics[width=0.29\textwidth]{file_name}
%	\end{center}
%	\vspace{-20pt}
%	\caption{}
%	\label{label:file_name}
%\end{wrapfigure}
%
%\begin{wrapfigure}{r}{0.45\textwidth}
%	\begin{center}
%		\includegraphics[width=0.29\textwidth]{manometer}
%	\end{center}
%	\caption{Aneroid sphygmomanometer with stethoscope (Medicalexpo, 2012).}
%	\label{label:manometer}
%\end{wrapfigure}
%
%
%\usepackage{subcaption}
%\begin{figure}[h]
%\begin{subfigure}{0.5\textwidth}
%\includegraphics[width=0.9\linewidth, height=5cm]{lion-logo} 
%\caption{Caption1}
%\label{fig:subim1}
%\end{subfigure}
%\begin{subfigure}{0.5\textwidth}
%\includegraphics[width=0.9\linewidth, height=5cm]{mesh}
%\caption{Caption 2}
%\label{fig:subim2}
%\end{subfigure}
%\caption{Caption for this figure with two images}
%\label{fig:image2}
%\end{figure}